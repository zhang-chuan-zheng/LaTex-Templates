\documentclass{Thesis}

\title{南京工业大学本科生毕业论文模板使用教程}{Nanjing Tech University undergraduate graduation thesis template tutorial}
\cover{2023}{我的专业}{我的班级}{我的姓名}{我的指导老师}{起讫时间}{06}


\begin{document}
	\makecover
	\pagenumbering{Roman}
	\phantomsection
	\addcontentsline{toc}{section}{摘要}
	\makecntitle
	\begin{cnabstract}
		本文讲解了如何使用南京工业大学本科生毕业论文的Thesis类文件,主要介绍tex、bib文件的书写规范,包括封面、题目、目录、图、表、公式、参考文献等。\\
		\vspace{13pt}
		
		\noindent\textbf{关键词:}关键词1 \quad 关键词2 \quad 关键词3 \quad 关键词4 \quad 关键词5
	\end{cnabstract}
	
	\newpage
	\phantomsection
	\addcontentsline{toc}{section}{Abstract}
	\makeentitle
	\begin{enabstract}
		This article explains how to use the Thesis documents of undergraduate dissertations of Nanjing University of Technology, mainly introducing the writing specifications of tex and bib files, including covers, titles, table of contents, figures, tables, formulas, references, etc.\\
		\vspace{13pt}
		
		\noindent\textbf{Key Word:}Figure;Table;Mathematical Formulas;Bibliography;Title
	\end{enabstract}
	
	\newpage
	\phantomsection
	\addcontentsline{toc}{section}{目录}
	\tableofcontents
	
	\newpage
	\pagenumbering{arabic}
	\section{封面}
	需要在导言区,也就是\verb|\begin{document}...\end{document}|之前,使用\verb|\title{}{}|和\\ \verb|\cover{}{}{}{}{}{}{}|两个命令。第一个命令是传入论文的中、英文名称,前者是中文,后者是英文。第二个命令需要输入六个参数,分别是年份、专业、班级、姓名、指导老师、起讫日期、完成月份。
	
	\section{数学公式}
	\subsection{行内公式}
	这是一个行内公式$E = mc^2$。还可以使用\( E= mc^2 \)
	\subsection{行间公式}
	\subsubsection{无序号公式}
	(1)使用\verb|\[   \]|
	\[
	E = mc^2 \label{f1}
	\]
	
	\ref{f1}
	
	(2)使用\verb|$$  $$|
	$$E = mc^2$$
	
	(3)使用\verb|\tag{}|手动编号
	\[
	E = mc^2 \tag{2-1}
	\]
	
	(4)带*的环境
	\begin{equation*}
		E = mc^2
	\end{equation*}
	
	\subsubsection{有序号的公式}
	(1)使用equation环境
	\begin{equation}
		E = mc^2
	\end{equation}
	
	(2)使用split环境
	\begin{equation}
		\begin{split}
			a &= b+c+d+e\\ 
			  &= f+g+h\\ 
			  &= i+j+k
		\end{split}
	\end{equation}

	(3)使用multline
	\begin{multline}
		a+b+c+d+e+f\\
		+g+h+i+j+k+l
	\end{multline}

	(4)使用gather
	\begin{gather}
		a_1 = b_1+c_1+d_1\\
		a_2 = b_2+c_2
	\end{gather}

	(5)使用align
	\begin{align}
		a_1 &= b_1+c_1+d_1\\
		a_2 &= b_2+c_2
	\end{align}

	\begin{align}
		a &= b+c+d+e\\ 
		  &= f+g+h\\ 
		  &= i+j+k
	\end{align}

	\begin{align}
		(\dfrac{\partial T}{\partial p})_S & = (\dfrac{\partial V}{\partial S})_p & (\dfrac{\partial S}{\partial V})_T & = (\dfrac{\partial p}{\partial T})_V \\
		(\dfrac{\partial S}{\partial p})_T & = -(\dfrac{\partial V}{\partial T})_p & (\dfrac{\partial T}{\partial V})_S & = -(\dfrac{\partial p}{\partial S})_V
	\end{align}
	
	(6)使用flalign
	\begin{flalign}
		(\dfrac{\partial T}{\partial p})_S & = (\dfrac{\partial V}{\partial S})_p & (\dfrac{\partial S}{\partial V})_T & = (\dfrac{\partial p}{\partial T})_V \\
		(\dfrac{\partial S}{\partial p})_T & = -(\dfrac{\partial V}{\partial T})_p & (\dfrac{\partial T}{\partial V})_S & = -(\dfrac{\partial p}{\partial S})_V
	\end{flalign}
	
	(7)使用alignat
	\begin{alignat}{2}
		a_1 &= b_1+c_1+d_1 &\quad& 1 \label{a1}\\
		a_2 &= b_2+c_2 & &2 \label{a2}
	\end{alignat}
	
	见式\ref{a1}、\ref{a2}
	
	(8)使用cases
	\begin{equation}
		y = 
		\begin{cases}
			x_1 + x_2\ ,\quad x_1<x_2\\
			x_1 - x_2\ ,\quad x_1>x_2
		\end{cases}
	\end{equation}
	
	\begin{equation*}
		y = \left\{
		\begin{split}
			&x_1 + x_2\ ,\quad x_1<x_2\\
			&x_1 - x_2\ ,\quad x_1>x_2
		\end{split}
		\right.
	\end{equation*}
	
	\[
	y = \left\{
	\begin{split}
		&x_1 + x_2\ ,\quad x_1<x_2\\
		&x_1 - x_2\ ,\quad x_1>x_2
	\end{split}
	\right.
	\]
	
	(9)共用编号
	\begin{equation}
		\begin{aligned}
			a_1 &= b_1+c_1+d_1\\
			a_2 &= b_2+c_2
		\end{aligned}
	\end{equation}

	\begin{equation}
		\begin{gathered}
			a_1 = b_1+c_1+d_1\\
			a_2 = b_2+c_2
		\end{gathered}
	\end{equation}

	\subsection{数组和矩阵}
	\subsubsection{使用array}
	\begin{equation}
		\mathbf{X} = \left(
		\begin{array}{cccc}
			x_{11}&x_{12}&\ldots & x_{1n}\\
			x_{21}&x_{22}&\ldots & x_{2n}\\
			\vdots&\vdots&\ddots & \vdots\\
			x_{n1}&x_{n2}&\ldots & x_{nn}
		\end{array}
		\right)
	\end{equation}	
	\subsubsection{使用matrix及其它包含界定符的环境}
	\begin{equation}
		\mathbf{X} =
		\begin{matrix}
			x_{11}&x_{12}&\ldots & x_{1n}\\
			x_{21}&x_{22}&\ldots & x_{2n}\\
			\vdots&\vdots&\ddots & \vdots\\
			x_{n1}&x_{n2}&\ldots & x_{nn}
		\end{matrix}
	\end{equation}
	\begin{equation}
		\mathbf{X} =
		\begin{pmatrix}
			x_{11}&x_{12}&\ldots & x_{1n}\\
			x_{21}&x_{22}&\ldots & x_{2n}\\
			\vdots&\vdots&\ddots & \vdots\\
			x_{n1}&x_{n2}&\ldots & x_{nn}
		\end{pmatrix}
	\end{equation}

	\begin{equation}
		\mathbf{X} =
		\begin{bmatrix}
			x_{11}&x_{12}&\ldots & x_{1n}\\
			x_{21}&x_{22}&\ldots & x_{2n}\\
			\vdots&\vdots&\ddots & \vdots\\
			x_{n1}&x_{n2}&\ldots & x_{nn}
		\end{bmatrix}
	\end{equation}
	\begin{equation}
		\mathbf{X} =
		\begin{Bmatrix}
			x_{11}&x_{12}&\ldots & x_{1n}\\
			x_{21}&x_{22}&\ldots & x_{2n}\\
			\vdots&\vdots&\ddots & \vdots\\
			x_{n1}&x_{n2}&\ldots & x_{nn}
		\end{Bmatrix}
	\end{equation}
	\begin{equation}
		\mathbf{X} =
		\begin{vmatrix}
			x_{11}&x_{12}&\ldots & x_{1n}\\
			x_{21}&x_{22}&\ldots & x_{2n}\\
			\vdots&\vdots&\ddots & \vdots\\
			x_{n1}&x_{n2}&\ldots & x_{nn}
		\end{vmatrix}
	\end{equation}
	\begin{equation}
		\mathbf{X} =
		\begin{Vmatrix}
			x_{11}&x_{12}&\ldots & x_{1n}\\
			x_{21}&x_{22}&\ldots & x_{2n}\\
			\vdots&\vdots&\ddots & \vdots\\
			x_{n1}&x_{n2}&\ldots & x_{nn}
		\end{Vmatrix}
	\end{equation}

	\subsection{字体}
	\subsubsection{常用字体命令}
	我有一个变量,$N$,但是后面N变成了常量,我们就要用到正体格式,$\rm N$.
	\[
	(\dfrac{\partial \mathit{T}}{\partial \mathit{p}})_\mathit{S}
	\]
	
	比如说我们在计算导数的时候,我们会使用
	\[
	v = \frac{\mathrm{d}x}{\mathrm{d}t}
	\]
	
	\paragraph{mathrm}$\mathrm{ABCDE123456}$
	\paragraph{mathit}$\mathit{ABCD123}$
	\paragraph{mathbf}$\mathbf{ABCD123}$ 
	\paragraph{mathbb}$\mathbb{ABCD}$
	\paragraph{mathcal}$\mathcal{ABCD}$
	
	\subsubsection{希腊正体}
	$\alpha$ $\mathrm{\alpha}$ $\upalpha$
	
	$$
	\begin{array}{cc|cc}
		斜体&正体&斜体&正体\\
		\alpha&\upalpha&\beta&\upbeta \\
		\mu&\upmu&\nu&\upnu
	\end{array}
	$$
	
	\section{列表}
	\subsection{无序列表}
	\begin{itemize}
		\item[+] 封面
		\item[+] 数学公式 \label{i1}
		\begin{itemize}
			\item 行内公式
			\item 行间公式
		\end{itemize}
		\item 列表
	\end{itemize}

	\ref{i1}
	
	\subsection{有序列表}
	\begin{enumerate}
		\item 封面 \label{it1}
		\item 数学公式
		\begin{itemize}
			\item 行内公式
			\item 行间公式
		\end{itemize}
		\item 列表
	\end{enumerate}
	
	\ref{it1}

	\section{图表}
	\subsection{三线表}
	\subsubsection{跨页表格}
	\begin{longtable}{p{0.2\textwidth}<{\centering}p{0.2\textwidth}<{\centering}p{0.2\textwidth}<{\centering}}
		\caption{跨页表格}
		\label{longtable1}\\
		\toprule[1.5pt]
		\makecell{如果标题\\比较长}&列2&列3\\
		\toprule[1.5pt]
		\endfirsthead
		
		\multicolumn{3}{r}{\small 续表\ref{longtable1}}\\
		\toprule[1.5pt]
		\makecell{如果标题\\比较长}&列2&列3\\
		\toprule[1.5pt]
		\endhead
		
		\toprule[1.5pt]
		\endfoot
		\toprule[1.5pt]
		\endlastfoot
		
		1&2&3\\
		4&5&6\\
		7&8&9\\
		10&11&12\\
		13&14&15\\
		16&17&18\\
		19&20&21\\
		22&23&24\\
		25&26&27\\
		28&29&30\\
		1&2&3\\
		4&5&6\\
		7&8&9\\
		10&11&12\\
		13&14&15\\
		16&17&18\\
		19&20&21\\
		22&23&24\\
		25&26&27\\
		28&29&30\\
		1&2&3\\
		4&5&6\\
		7&8&9\\
		10&11&12\\
		13&14&15\\
		16&17&18\\
		19&20&21\\
		22&23&24\\
		25&26&27\\
		28&29&30\\
		1&2&3\\
		4&5&6\\
		7&8&9\\
		10&11&12\\
		13&14&15\\
		16&17&18\\
		19&20&21\\
		22&23&24\\
		25&26&27\\
		28&29&30\\
		1&2&3\\
		4&5&6\\
		7&8&9\\
		10&11&12\\
		13&14&15\\
		16&17&18\\
		19&20&21\\
		22&23&24\\
		25&26&27\\
		28&29&30\\
		1&2&3\\
		4&5&6\\
		7&8&9\\
		10&11&12\\
		13&14&15\\
		16&17&18\\
		19&20&21\\
		22&23&24\\
		25&26&27\\
		28&29&30\\
		1&2&3\\
		4&5&6\\
		7&8&9\\
		10&11&12\\
		13&14&15\\
		16&17&18\\
		19&20&21\\
		22&23&24\\
		25&26&27\\
		28&29&30\\
		1&2&3\\
		4&5&6\\
		7&8&9\\
		10&11&12\\
		13&14&15\\
		16&17&18\\
		19&20&21\\
		22&23&24\\
		25&26&27\\
		28&29&30\\
		1&2&3\\
		4&5&6\\
		7&8&9\\
		10&11&12\\
		13&14&15\\
		16&17&18\\
		19&20&21\\
		22&23&24\\
		25&26&27\\
		28&29&30\\
	\end{longtable}

	\subsubsection{普通表格}
	\begin{tabular}{p{0.4\textwidth}<{\centering}p{0.4\textwidth}<{\centering}}
		\toprule[1.5pt]
		列1 & 列2\\
		\toprule[1.5pt]
		1&2\\
		3&4\\
		5&6\\
		\toprule[1.5pt]
	\end{tabular}

	\begin{table}[htbp]
		\renewcommand\arraystretch{1}
		\centering
		\setlength{\abovecaptionskip}{10pt}
		\setlength{\belowcaptionskip}{0pt}
		\caption{这是一个普通表格}
		\label{table1}
		\begin{tabular}{p{0.4\textwidth}<{\centering}p{0.4\textwidth}<{\centering}}
			\toprule[1.5pt]
			列1 & 列2\\
			\toprule[1.5pt]
			1&2\\
			3&4\\
			5&6\\
			\toprule[1.5pt]
		\end{tabular}
	\end{table}

	见表\ref{table1},或者说见表\eqref{table1}
	
	\begin{table}[H]
		\renewcommand\arraystretch{1}
		\centering
		\setlength{\abovecaptionskip}{10pt}
		\setlength{\belowcaptionskip}{0pt}
		\caption{这是一个普通表格}
		\label{table2}
		\begin{tabular}{p{0.4\textwidth}<{\centering}p{0.4\textwidth}<{\centering}}
			\toprule[1.5pt]
			列1 & 列2\\
			\toprule[1.5pt]
			1&\multirow{2}{*}{6}\\
			3&\\
			5&6\\
			\toprule[1.5pt]
		\end{tabular}
	\end{table}
	
	\subsection{图}
	\begin{figure}[H]
		\centering
		\includegraphics[width=0.5\textwidth]{pictures/bilibiliQRcode.jpg}
		\caption{哔哩哔哩二维码}
		\label{fig1}
	\end{figure}
	
	\begin{figure}[htbp]
		\centering
		\subfigure[]{
			\includegraphics[width=0.3\textwidth]{pictures/bilibiliQRcode.jpg}
			\label{subfi1}
		}
		\subfigure[]{
			\includegraphics[width=0.3\textwidth]{pictures/zhihuQRcode.png}
			\label{subfi2}
		}
		
		\subfigure[]{
			\includegraphics[width=0.3\textwidth]{pictures/bilibiliQRcode.jpg}
			\label{subfi3}
		}
		\subfigure[]{
			\includegraphics[width=0.3\textwidth]{pictures/zhihuQRcode.png}
			\label{subfi4}
		}
		\caption{二维码;\subref{subfi1} 哔哩哔哩二维码;\subref{subfi2}知乎二维码}
	\end{figure}

	\ref{subfi1}
	
	\newpage
	\phantomsection
	\addcontentsline{toc}{section}{参考文献}
	\bibliographystyle{gbt7714-numerical}
	\bibliography{Thesis}

	
	\newpage
	\phantomsection
	\addcontentsline{toc}{section}{致谢}
	\section*{致谢 \markboth{致谢}{}}
	我想要输入一个空{ }格,一个空 \quad 格,还可以使用\qquad 生成空格;再降一个\,或者\ 或者\;或者说。
	我想输入一个下角标,可以使用数学环境$\rm D_s$,如果是上角标$\rm 3^2$,我们可以这样$^{[1]}$。
	参考文献使用\verb|\cite{}|命令。
\end{document}